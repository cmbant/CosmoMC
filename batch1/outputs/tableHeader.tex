\title{\textit{Planck} 2013 Results: Cosmological Parameter Tables}


%\titlerunning{Planck Cosmological Parameter Tables}
\maketitle
\begin{abstract}
These tables summarize the results of \textit{Planck} nominal mission parameter estimation exploration results. They include \textit{Planck}  data in combination with WMAP polarization, \textit{Planck} lensing, and high-$l$ CMB experiments, as well as additional non-CMB data as detailed in the main parameter papers.
\end{abstract}

\newpage
\section{Introduction}

The tables are arranged grouped firstly by cosmological model, and then by data combination. The name tags match those of the full chains also provided on the PLA. They all start with {\tt base} to denote the baseline model, followed by the parameter tags of any additional parameters that are also varied (as defined in the parameter paper). Data combination tags are as follows (see the parameters paper for full description and references):

\begin{tabular} { l   l  }
Data tag & Data used\\
\hline
{\tt planck}         & high-$l$ \textit{Planck}  temperature ({\tt CamSpec}, $50\le l\le 2500$) \\
{\tt lowl }          & low-$l$ \textit{Planck}  temperature ($2\le l \le 49$)  \\
{\tt lensing}        & \textit{Planck}  lensing power spectrum reconstruction \\
{\tt lowLike}        & low-$l$ WMAP 9 polarization \\
{\tt tauprior}       & A Gaussian prior on the optical depth, $\tau = 0.09 \pm 0.013$ \\
{\tt BAO}            & Baryon oscillation data from DR7, DR9 and and 6DF \\
{\tt SNLS}           & Supernova data from the Supernova Legacy Survey \\
{\tt Union2}         & Supernova data from the Union compilation \\
{\tt HST}            & Hubble parameter constraint from HST (Riess et al) \\
{\tt WMAP}           & The full WMAP (temperature and polarization) 9 year data \\
\hline
\end{tabular}
\vskip 1cm

Data likelihoods are either included when running the chains, or by importance sampling. Data combinations that are added by importance sampling appear
at the end of the list, following the {\tt post{\textunderscore}} tag. WMAP9 chains are run from the WMAP9 likelihood code with the same baseline assumptions as \textit{Planck}, and hence
may different slightly from those available on Lambda (e.g. the baseline model has non-zero neutrino mass). Note that the best fits are merely examples of parameter combinations that fit the data well, due to parameter degeneracies there may be other combinations of parameters that fit the data nearly equally well.

Beneath each table is the minus log Likelihood $\chi^2_{\rm eff}$ for each best fit model, and also the contributions coming from each separate part of the likelihood. The $R-1$ value is also given, which measures the convergence of the sampling chains, with small values being better converged. The sampling uncertainty on quoted mean values are typically of order $R-1$ in units of the standard deviation.

%Files are provided in the format produced by the CosmoMC code, available from http://cosmologist.info/cosmomc.

\newpage
